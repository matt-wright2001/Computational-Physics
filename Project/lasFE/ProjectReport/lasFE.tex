\documentclass[aps, prl, twocolumn, superscriptaddress, showpacs]{revtex4-1}

\usepackage{graphicx}
\usepackage{mathrsfs}
\usepackage{verbatim}
\usepackage{ulem}
\usepackage{hyperref}
\usepackage{gensymb}
\usepackage{xcolor}

\newcommand{\msstate}{Department of Physics and Astronomy, Mississippi State University, Mississippi State, MS 39762, USA}
\newcommand{\icet}{Institute for Clean Energy Technology, Mississippi State University, Mississippi State, MS 39762, USA}

\begin{document}
\title{lasFE: Lognormal Fitting of Laser Aerosol Spectrometer data for Filtration Efficiency Analysis}

\author{M.~S.~Wright}
\email{wright@icet.msstate.edu}
\affiliation{\icet}
\affiliation{\msstate}

\begin{abstract}
	\verb|lasFE| facilitates the analysis of particle-size distribution (PSD) data collected using a TSI model 3340A Aerosol Spectrometer (LAS) to determine number-collection filtration efficiency (FE). The software is designed to iterativly fit aerosol particle-size data collected upstream and downstream of a filter to a lognormal distribution to characterize aerosol properties. While this code was written for purposes of \verb|PH 6433 (Computational Physics)|, software development continues for nuclear-grade HEPA filter qualification and performance evaluation. This early preview of \verb|lasFE| features data importation and preprocessing, nonlinear fitting, numerical integration, error analysis, and plotting functionalities. Ongoing development and issues are discussed.
\end{abstract}

\maketitle

\section{Introduction}
\section{Software Quality Assurance}
\section{Statistical Methods}
\section{Results}
\section{Discussion}
\section{Conclusion}


\bibliographystyle{apsrev4-1}
\bibliography{lasFE}

\end{document}
