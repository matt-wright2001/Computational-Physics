\documentclass[aps, prl, twocolumn, showpacs]{revtex4-2}

\usepackage{graphicx}
\usepackage[margin=0.75in]{geometry}
\usepackage{mathrsfs}
\usepackage{verbatim}
\usepackage{ulem}
\usepackage{hyperref}
\usepackage{gensymb}
\usepackage{xcolor}

\newcommand{\msstate}{Department of Physics and Astronomy, Mississippi State University, Mississippi State, MS 39762, USA}
\newcommand{\icet}{Institute for Clean Energy Technology, Mississippi State University, Mississippi State, MS 39762, USA}

\begin{document}
\title{lasFE: Lognormal Fitting of Laser Aerosol Spectrometer data for Filtration-Efficiency Analysis}

\author{M.~S.~Wright}
\email{wright@icet.msstate.edu}
\affiliation{\icet}
\affiliation{\msstate}

\begin{abstract}
	\verb|lasFE| facilitates the analysis of particle-size distribution (PSD) data collected using a TSI model 3340A Laser Aerosol Spectrometer (LAS) to determine number-collection filtration efficiency (FE) by comparing aerosol properties upstream and downstream of a filter. The software is designed to iterativly fit aerosol particle-size data to a lognormal distribution to determine the distribution's geometric mean and geometeric standard deviation. While this code was written for purposes of \verb|PH 6433 (Computational Physics)|, software development continues for nuclear-grade HEPA filter qualification or performance evaluation. This early preview of \verb|lasFE| features data importation and preprocessing, nonlinear fitting, numerical integration, error analysis, and plotting functionalities. Ongoing development and issues are discussed.
\end{abstract}

\maketitle

\section{Introduction}
The LAS, an optical spectrometer using an intracavity laser to measure an aerosol sample's particulate size and number concentration, is commonly used to test nuclear-grade High Efficiency Particulate Air (HEPA) filters at the Institute for Clean Energy Technology (ICET).  \verb|lasFE| is software developed under ICET's \textit{American Society for Mechanical Engineers, Nuclear Quality Assurance (NQA-1)} program to perform a rigorous analysis of data produced with a TSI-manufactured LAS (model 3340A)\cite{tsi_las3340a}. For my \verb|PH 6433| final project, I present a prerelease of \verb|lasFE| highlighting a failed nonlinear-fit procedure and an actionable path forward controlled by \textit{ICET-QA-036, Software Control}\cite{qa-036}.



\section{Software Quality Assurance}
\section{Statistical Methods}
\section{Results}
\section{Discussion}
\section{Conclusion}


\bibliographystyle{apsrev4-1}
\bibliography{lasFE}

\end{document}
